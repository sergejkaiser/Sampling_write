The effects of age in 7 categories, income level in 3 categories and sex on having a stable general practitioner or not were studied by running a logistic regression with dummy coding. There are two analyses one taking the survey design into account and the other not. There were 436 units with variable values not available, they were deleted pairwise, so in total 8128 units of observations were used.

Here we will compare a regression equation where none of the design features were taken into account and a one where all of them (finite population correction, stratification, clustering and weighting) are taken into account. Tables below show the estimates for log odds and standard errors for all the variables.

\begin{table}[H]
\centering
\adjustbox{max height=\dimexpr\textheight-5cm,
            max width=\textwidth, center}{
\begin{tabular}{llllll}
	\toprule 
	& Level & Estimate & S.E. & t value & $Pr > |t|$\\
	\midrule 
Intercept & & -3.13 & 0.22 & -14.16 & 0.0 \\
SEX 	& 1 &	0.44 &	0.08 &	5.78 &	0.0 \\
AGE7 	& 1 &	1.05 &	0.22 &	4.69 &	0.0 \\
AGE7 	& 2 &	1.33 &	0.21 &	6.21 &	0.0 \\
AGE7 	& 3 &	1.14 &	0.22 &	5.31 &	0.0 \\
AGE7 	& 4 &	0.72 &	0.23 &	3.19 &	0.0 \\
AGE7 	& 5 &	0.67 &	0.23 &	2.88 &	0.0 \\
AGE7 	& 6 &	0.26 &	0.24 &	1.08 &	0.3 \\
FA3 	& 1 &	-0.19 &	0.10 &	-1.84 &	0.07 \\
FA3 	& 2 &	-0.45 &	0.11 &	-3.96 &	0.0 \\
\bottomrule
\multicolumn{3}{l}{\footnotesize }
\end{tabular}}
	\caption{No consideration for survey design}
	\label{nodesign}
\end{table}





\begin{table}[H]
\centering
\adjustbox{max height=\dimexpr\textheight-5cm,
            max width=\textwidth, center}{
\begin{tabular}{llllll}
	\toprule 
	& Level & Estimate & S.E. & t value & $Pr > |t|$\\
	\midrule 
Intercept & &-3.45 & 0.49 & -7.08 & 0.0 \\
SEX &	1 &	0.32 &	0.10 &	3.11 &	0.00 \\
AGE7 &	1 &	1.03 &	0.48 &	2.14 &	0.03 \\
AGE7 &	2 &	1.16 &	0.47 &	2.45 &	0.01 \\
AGE7 &	3 &	1.11 &	0.48 &	2.32 &	0.02 \\
AGE7 &	4 &	0.60 &	0.48 &	1.25 &	0.21 \\
AGE7 &	5 &	0.20 &	0.49 &	0.41 &	0.68 \\
AGE7 & 	6 &	0.24 &	0.53 &	0.46 &	0.64 \\
FA3 &	1 &	-0.12 &	0.19 &	-0.63 &	0.53 \\
FA3 &	2 &	-0.41 &	0.21 &	-1.98 &	0.05 \\
\bottomrule
\multicolumn{3}{l}{\footnotesize }
\end{tabular}}
	\caption{Consideration for survey design}
	\label{withdesign}
\end{table}





Parameter estimates vary between the two designs. The tables show that all of the standard errors went up, doubled for most of the parameters when the survey design was taken into account. As dummy coding was used estimate for the intercept shows the log odds for the reference category (female, older than 75 and highest income group) and the parameter estimates show the results compared to the reference of that category, in this case odds of having a stable GP is 4\% of not having one ($exp(-3.45)$). Men have higher odds of having a GP than women. Odds of having a GP gets lower as persons get older and poorer.

Below are Wald tests for parameters. In both cases age and sex effects are significantly different than zero, whereas when design is taken into account income effect is not significantly different than zero at \% 5 level, but it is significantly different than zero when design is not taken into account. 

\begin{table}[H]
\centering
\adjustbox{max height=\dimexpr\textheight-5cm,
            max width=\textwidth, center}{
\begin{tabular}{llll}
	\toprule 
	Effect  & DF & Wald Chisq & Pr $>$ ChiSq \\
	\midrule 
	SEX &	1 & 	33.36 & 	0.0 \\
	AGE7 &	6 &	89.80 &	0.0 \\
	FA3 &	2 &	17.08 & 	0.0 \\
\bottomrule
\multicolumn{4}{l}{\footnotesize }
\end{tabular}}
	\caption{Type 3 Analysis of Effects w/out design}
	\label{withdesign}
\end{table}





\begin{table}[H]
\centering
\adjustbox{max height=\dimexpr\textheight-5cm,
            max width=\textwidth, center}{
\begin{tabular}{llll}
	\toprule 
	Effect  & DF & Wald Chisq & Pr $>$ ChiSq \\
	\midrule 
SEX 	& 1 	& 9.65 &	0.0\\
AGE7 	& 6 	& 38.87 &	0.0\\
FA3 	& 2 &	5.25 &	0.07\\
\bottomrule
\multicolumn{4}{l}{\footnotesize }
\end{tabular}}
	\caption{Type 3 Analysis of Effects w/ design}
	\label{withdesign}
\end{table}

