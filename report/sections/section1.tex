In the following we are going to first summarize "The problem of nonrespone in sample survey" by Hurwitz and Hansen (2004) in combination with "Multiphase Sampling in Nonresponse Problems" by Srinath (1971). 
Hurwitz and Hansen describe sampling procedures to obtain a cost optimal allocation of survey mail and follow up face to face interviews. Substaintially the authors argue that combining both one may obtain the advantages from both methods, thus low cost of mail with higher repsonse and infromation conerning the representativity of the sample to the population. 
- first point sample size is only one factor of sampling error
- shows based on a variance formula that same reliability may arrise from various sample sizes
- shows optimal number of mails and in person interviews for a desired average error can be calculated if an approximate response rate is available.
- if the response rate is unkown the authors showed that the optimal number of mails and in person interviews can be obtained for a given range of reponse rates.
-  further highlighted two alternatives plans to achieve minimum cost for a desired avg. level of precision (for a response rate of 100\% assuming similarity between respondents and non respondents ).
- Alternative 1 initially send out 1000 mails and follow up on all the non responses regardless of the non response rate.
  Result: Higher cost than optimal combination, for response rate >30, cost increase from 10 to 24 percent.
- Alternative 2 preferable if approximate reponse is unkown, specify maximum number of responss sent out regardless of non response. Second step after sample return and response rate are known, set the number of follow up interviews such that desired level of precision is achieved.
- Striking result for alternative 2 even though response rate not known, for a given level of precision may be reached for low level of cost (not much higher than optimal level of cost). 
- Critique: Cost depend on approx. rate of response. This may be a problem in practice that either the targeted avg. sampling error is missed or cost budget is out of control. 
- Alternative 2 initial step unrealistic. Why not set sample size such that for a upper range of non response a desired level of precision is obtained. 
- It might be more cost saving strategy to combine several attempts of mail survey with a final interview follow up. 
- Further short comming is that the authors only focus on non response. The reliability of an sample depends as well on the quality of the respondents answers. We could imagine that simple extension to include a quality of response trade off, could alter the balance. 
K.P. Srinath extends the sampling plan of Hansen and Hurwitz. The authors contribution is that he proposes to use subsampling fractions, which vary in response to sample non response. The authors derrives expressions for the variance of the estimator which do not depend on the unkown sampling variance and the subsampling fractions are adjusted to ensure a fixed precision.  

\endinput
