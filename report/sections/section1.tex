In the following we are going first to summarise "The problem of nonresponse in sample survey" by Hurwitz and Hansen (2004) in combination with "Multiphase Sampling in Nonresponse Problems" by Srinath (1971). \par
The first paper discusses how to choose the optimal subsample after an initial non-response of contacted persons in a sample survey. The first contact attempt is done with via mail, and the random subsample may be tried to reach via mail. 
 Substantially the authors argue that one may achieve a cost reduction using the mail and combining it with the follow-up personal interview to ensure a high response rate and the representativity of the sample to the population. \par 
First, the authors derived under the assumption that an approximate response rate is known simple expressions for the optimal number of mails and follow-up interviews, which minimising the cost at a desired average sampling error. Their optimal number of mails depends on factors as the population size, the population variance and the response rate. The subsample of interviews is than proportional to the inverse of the constant, which contains the ratio of cost and response rates. \par
Further,  if the response rate is unknown, the authors highlighted two alternatives plans to achieve minimum cost for the desired average level of precision under the assumption of a response rate of 100\% . The first plan of the authors is to initially send out 1000 mails and follow up on all the non-responses regardless of the non-response rate. They find that
the result is higher cost than optimal combination for response rate of high than 30 percent the cost increase is in the range of 10 to 24 percent. \par
A second plan the authors suggest to improve cost efficiency is to specify a maximum number of responses sent out regardless of the non-response. In a second step after the sample returned and response rate are known,  the number of follow-up interviews is determined such that desired level of precision is achieved. 
The authors find that the second plan without using information about response rate for a given level of precision may be reached at a slightly higher level of cost compared to the optimal values given the response rate.  \par
The second paper of K.P. Srinath suggests an interesting extension to the sampling plan of Hansen and Hurwitz. The authors contribution is that he proposes to use subsampling fractions, which vary in response to sample non-response. The authors derive expressions for the variance of the estimator which do not depend on the unknown sampling variance, and the subsampling fractions are adjusted to ensure a fixed precision.  
\par
We agree with the authors that combining data collection methods with an initially more inexpensive with more expensive increase the response yields a cost saving and response increasing method.  Further, we agree with the authors example that an increase of sample size is not a sole solution to reliability, as other design factors may be more important for the reliability.  \par 
Further, we agree with the extension of Srinath to adjust the follow up to non-response rates, as we think that using this information one can obtain a smaller sample for a given precision.  \par
We disagree with Hurwitz and Hansen as we do not think that the first plan is a good plan for two reasons. It is not cost efficient and second taking a sample of the non-respondents may yield the same precision and may reduce cost. 
A critique, which we share with the literature,  is that the cost savings of the method are not high if the expected cost of an interview is not much higher than the expected cost of mail survey  (\cite{durbin1953}). 
Further, we think that the second plans described in Hurwitz and Hansen are too conservative in assuming that no response rate is available. We think that it may be more cost efficient to determine an initial sample size under a conservative estimate of the upper range of non-response. 
Also, the authors of the first paper assume that the response is one hundred percent after the interview. We think that this assumption is to our opinion questionable. 
The desired sampling error thus may not be achieved, and a larger sample and subsample may be necessary.\par
Moreover, we think that a cost saving alternative to both plans of Hurwitz and Hansen is to use several attempts of mail survey with a final interview follow up. This has been an also suggest in the literature by  \cite{el1956} and is also discussed in \cite{Srinath}. 
\par
Another shortcoming is that both papers only focus on non-response. 
However, we think that the reliability of a sample depends as well on the quality of the respondents answers. 
We think that including the quality dimension to the model could yield important changes to the optimal allocation of data collection methods. 
  \par Further we think that the sampling plans in both papers could be improved regarding field efficiency by using prior information to satisfy the sample. A higher field efficiency may then allow to sample more units and therefore improved precision for a given level of cost. 
\endinput
